\documentclass{ook}
\usepackage[left=1in,right=1in,top=1in,bottom=1in]{geometry}
\usepackage{amsmath}
\usepackage{amsfonts}
\usepackage{amssymb}
\usepackage{booktabs}
\usepackage{enumerate}
\title{CSC 203 {-} Introduction to computer systems}
\author{Abdulqadir Ahmad}
\begin{document}
\maketitle
\tableofcontents
\section{Computer Systems}
Computer systems in this course comprises of hardwares and softwares.

\chapter{Number system}

\section{Positional number system}

In positional number system, each digit in a number has a weight. Weight in this case is the base number raise to the power of the associated position.

The power starts at 0 on the first number before the radix or base point (decimal point when dealing with decimal numbers {-} base 10). It increases to the left and decreases to the right.

\section{number base and number base conversion}

We will cover four number bases:

\begin{enumerate}
	\item Binary
	\item Octal
	\item Decimal
	\item Hexadecimal
\end{enumerate}

After stripping all the complexities in a computer, it all boils down to zeros and ones; I.E. binary. All the other number bases are meant for human consuption or usage. For example, humans are more comfortable using decimal and it's easier to represent very large numbers in hexa decimal because it takes less space.

\subsection{Decimal {-} base 10}

This base has 10 digits starting from 0 to 9. As stated earlier, humans are more comfortable with this base.

\subsection{Conversion from decimal to other bases}

To convert a number in base 10 to any other base, you can use continuous division and multiplication. Continuous division for the numbers before the decimal point, and continuous multiplication for the numbers aftter the decimal point.

The number in the target can be gotten by writing the remainders from the last to the first for the numbers before the radix point and from the first remainder to the last for the number after the radix point.

\subsection{Conversion from other bases to decimal}

To convert a number from any base to decimal, you can use the polinomail expansion method.
Polinomial expansion has to do with multiplying each digit in a number by it's weight and then adding the result.

\subsection{Binary {-} base 2}
Binary has two digits, zero and one. This is the base that computers works with.

\subsection{Conversion from binary to octal and hexa decimal}

The conversion method in this case is known as the grouping method. The highest number in octal is 7 (which is 111 in binary), and the highest number in hexadecimal is F (which is 1111 in binary). To convert a number in binary to either octal or hexadecimal, you can group the bits in 3s for octal and 4s for hexadecimal and replace each group with it's corresponding value in the target base. To be able to use this methd, you should be able to count in binary.

\subsection{Conversion from hexadecimal and octal to binary}

I call this method the unpacking or ungrouping method because this is the reverse of the grouping method. You take each digit and replace it with it's corresponding value in binary. Each octal digit takes up 3 bits while each hexadecimal digit takes up 4 bits.

\subsection{octal {-} base 8}

There are 8 numbers in octal, starting from 0 to 7. Octal can be used in representing bytes in a computer.

\subsection{Hexadecimal {-} base 16}

There are 16 numbers in hexadecimal, starting from 0 to F, where 10 in base 10 is A up to \({15}_{10} = F\). Hexadecimal can be used to represent memory in a computer or very large numbers.

\subsection{Conversion between octal and hexadecimal}
Conversion between octal and hexadecimal can be done through binary or decimal.

\chapter{Data representation}

\section{Number representation}

\subsection{Unsigned integers}

\subsection{signed integers}

\subsection{One's complement}

\subsection{Two's complement}

\section{Floating point representation}

sign -> exponent -> mantisa

\subsection{Single precision}

\subsection{Double precision}

\subsection{Quadriple precision}
\end{document}
