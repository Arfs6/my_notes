\documentclass{book}
\usepackage[left=1in,right=1in,top=1in,bottom=1in]{geometry}
\usepackage{amsmath}
\usepackage{amsfonts}
\usepackage{amssymb}
\usepackage{booktabs}
\usepackage{enumerate}
\title{MTH 202 {-} Differential Equations}
\author{Abdulqadir Ahmad}
\begin{document}
\maketitle
\tableofcontents

\chapter{First order differential equations}
\section{Linear differential equations}
In order to be able to solve a linear differnetial equation, it most be in the form:
\[\frac{dy}{dx} + p(x) y = g(x)\]
Where both \(p(x)\) and \(g(x)\) are both continuous functions.

We'll start by finding a function \(\mu(x)\) (integrating factor) such that when the differential equation is multiplied through by it, the left side of the equality sign turns into a derivative of a product of 2 functions.

\[\mu(x) = e^{\int p(x)}\]

Multiplying through by \(\mu(x)\), we'll get:
\[\mu(x) y^'	+ \mu(x) p(x) y = g(x) \mu(x)\]

Recall that \(\mu(x) = e^{\int p(x)}\).
\[\therefore \mu^'(x) = \mu(x) p(x)\]

Substitute for \(\mu^'(x)\):
\[\mu(x) y^'(x) + \mu^'(x) y(x) = g(x) \mu(x)\]

If you look at the left hand side carefully, it's just the derivitive of \(y (x)\times \mu(x)\).

Let's change the left side of the equation to be the derivitive:
\[{(y(x) \cdot \mu(x))}^' = g(x) \mu(x)\]

To find \(y(x)\), we'll need to integrate both sides and solve for \(y(x)\):
\[y(x) = \frac{\int (g(x) \mu(x)) + c}{\mu(x)}\]

We could remove the integration constant because after integrating \(\int (\mu(x) g(x))\), another integration constant will immerge and we can merge the two.

\section{Separable differential equations}
This are non linear differential equations in the form:
\[N(y) \frac{dy}{dx} = M(x)\]

In order to solve this kind of equations, we can start by integrating both sides and hopefully be able to solve for y.
\[\int N(y) \frac{dy}{dx} {dx} = \int M(x) {dx}\]

Recall that \(y = y(x)\).

So if we let \(u = y\), \[\frac{du}{dx} = \frac{dy}{dx} \rightarrow {du} = \frac{dy}{dx} \cdot {dx}\]

We can substitute \(u\) and \({du}\):
\[\int N(u) {du} = \int M(x) {dx}\]

:
\[\int N(u) {du} = \int M(x) {dx}\]

This is the mathematical way for solving a separable differential equation.

Going back to our original equation, if we assume we can multiply through by \({dx}\), we'll have:
\[N(y) {dy} = M(x) {dx}\]

After integrating both sides by their corresponding term, we'll get the same solution as the mathematical method but with different variables:
\[\int N(y) {dy} = \int M(x) {dx}\]

\subsection{Solved questions}
\subsubsection{Question}

\[\]

\section{Exact equations}
Exact equations are the form:
\[f(x, y) = M(x, y) + N(x, y) {dy}/{dx} = 0\]

If a function \(f(x, y) = 0\), then \(\int f(x, y) = c\).

Let \(\int f(x, y) = \psi(x, y)\).

Using multiple variable chaine rule (\(f^'(x, y(x)) = \frac{\partial f}{\partial x} + \frac{\partial f}{\partial y} \cdot \frac{dy}{dx} \)) we can find \(f\) by deffernetiating \(\psi\).

Let \(\psi_x = \frac{\partial \psi}{\partial x}\)
and \(\psi_y = \frac{\partial \psi}{\partial y}\).

\[\therefore f(x, y) = \psi^'(x, y) = \psi_x + \psi_y {dy}/{dx} = M(x, y) + N(x, y) {dy}/{dx}\]

\subsection{Testing for exact equations}
\[M_y = N_x\]

No need to prove this.

I think it will be true as long as \(\psi\) exist.

\section{Bernoulli differential equation}
Bernoulli differential equations are usually in the form:
\[y^' + p(x) y = q(x) y^n\]

Where p and q are continuous functions and n is a real number.

The way of solving Bernoulli equations has to do with expressing the function as a linear function.

Let's start by dividing the equation through by \(y^n\):
\[y^{-n} y^' + p(x) y^{1 - n} = q(x)\]

Let \(v = y^{1 - n}\).
\[v^' = (1 - n)y^{-n} y^' \rightarrow \frac{1}{1 - n} v^' = y^{-n} y^'\]

Substituting both v and \(v^'\), we'll get:
\[\frac{1}{1 - n} v^' + p(x)v = q(x)\]

The above is a linear first order differential equation we can solve.

\section{Substitutions}
In Bernoulli's differential equations, we substituted \(v = y^{1-n}\). There are other substitutions that we can make to  make it possible to solve differential equations that would be difficult to solve otherwise.

\subsection{Homogeneous equations}
Homogeneous equations are in the form:
\[y^' = f(y/x)\]

First order differential equations that can be written in this form are known as homogeneous differential equations.s.
