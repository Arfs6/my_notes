\documentclass{book}
\usepackage[left=1in,right=1in,top=1in,bottom=1in]{geometry}
\usepackage{amsmath}
\usepackage{amsfonts}
\usepackage{amssymb}
\usepackage{booktabs}
\usepackage{enumerate}
\title{MTH 205 {-} Linear Algebra II}
\author{Abdulqadir Ahmad}
\begin{document}
\maketitle
\tableofcontents

\chapter{Systems of linear equations}

\section{Linear equation}

A linear equation over a field k can be represented by:

\[a_1 x_1 + a_2 x_2 + \cdots + a_n x_n = b\]

Where:

\[i, a_i, b \in k\]

\(x_i\) are unknown variables.

The scalars \(a_i\) are called the coefficient of \(x_i\) respectively and \(b\) is called the constant term of the linear equation.

\section{Solution of a linear equation}

A set of values \(c_i\) is said to be the solution of a linear equation if the linear equation is true when substituting \(c_i\) with \(x_i\).

\[a_1 c_1 + a_2 c_2 + \cdots + a_n c_n = b\]

Where \[c_1, c_2, c_n \in c_i\]

Solutions of linear equations can be categorised into 3:

\begin{enumerate}[a.]
	\item One of the coefficient of \(x_i\) is not equal to 0. Then we can write the solution as:

		\[a_1 x_1 = b - a_2 x_2 + \cdots + a_n x_n \rightarrow x_1 = 1/a_1 \times (b - a_2 x_2 \cdots + a_n x_n)\]

		In this case, we say that the linear equation has many solutions.

	\item All the coefficient of \(x_i\) are 0, but the constant is not 0.

		\[0 x_1 + 0 x_2 + \cdots + 0 x_n = b\]

		where \(b \ne 0\).

		In this case, we say the linear equation is not consistent for there is no solution.

	\item Both the coefficient of \(x_i\) and b are 0.

		\[0 x_1 + 0 x_2 + \cdots + 0 x_n = 0\]

		In this case, the solution of the linear equation can be any set of values in field K.
\end{enumerate}

\section{System of linear equations}

A system of linear equations is a set of m linear equations with n unknowns. For example:

\[a_{11} x_1 + a_{12} x_2 + \cdots + a_{1n} x_n = b_1\]
\[a_{21} x_1 + a_{22} x_2 + \cdots + a_{2n} x_n = b_2\]
\ldots
\[a_{m1} x_1 + a_{m2} x_2 + \cdots + a_{mn} x_n = b_m\]

where

\[a_{ij}, b_i \in K\]

If each \(b_i = 0\), then the system is said to be homogeneous otherwise, the system is non homogeneous.

\section{Solution of a system of linear equation}
A set of values \(c_i = \{c_1, c_2, \cdots c_n\}\) is a particular solution of a system if it satisfies all the linear equations.

The set of all particular solutions is termed the solution set or the general solution.

A homogeneous system always have a solution, I.E. the zero or trivial solution. Any other solution is called a non-trivial solution.

\subsection{Theorum}

Let C be a particular solution of a non-homogeneous system, and let A be the general solution of it's associated homogeneous system, then:

\[c + A = \{c + a \forall a in A\}\]

is the general solution of the aforementioned non-homogeneous system.

\section{Gauss elimination method}

This method involves the elimination of unknowns from succeeding equations by performing two operations:

\begin{enumerate}[I.]
	\item Interchange the equations so that first coefficient of the first isn't 0: \(a_{11} \ne 0\)
	\item For each row in a column, apply the equation:

		\[R_i = -A_{i1}R1 + a_{11}R_i\]

		Where:

		\(A_{ij}\) represents coefficients using matrix element identification.

		\(R_i\) represents a row.

		The goal is to have zeros below the \"diagonal\" of the system. You can achieve this by moving \(a_{ij}\) along the columns.
\end{enumerate}

\subsection{Coefficient and augmented matrix}

An augmented matrix is gotten by writing each equation in a system as a row in a matrix. The unknown variables removed nd a vertical line should be added to represent the equality sign.

A coefficient matrix is an augmented matrix without the constants.


\subsection{Echelon form}

\begin{enumerate}[I.]
	\item The first coefficient of the first row is not equal to zero: \(a_{11} \ne 0\)
	\item All coefficient below the leading diagonal are zeros.
\end{enumerate}

\subsection{Question}

What condition(s) must be placed on e, f and g so that the following system in unknowns x, y and z has:

\begin{enumerate}[(i)]
	\item No solution.
	\item More than one solution.
	\item A unique solution.
\end{enumerate}

\[x + 2y - 3z = e\]
\[2x - y + 4z = f\]
\[4x + 3y - 2z = g\]

\subsection*{Solution}
\[-2 R_1 + R_2 \rightarrow R_2\]
\[-4 R_1 + R_3 \rightarrow R_3\]

\[x + 2y - 3z = e\]
\[-5y + 10z = f - 2e\]
\[-5y + 10z = g - 4e\]

\[5 R_2 + (-5) \cdot R_3 \rightarrow R_3\]

\[x + 2y - 3z = e\]
\[-5y + 10z = f - 2e\]
\[0 = -5g + 10e + 5f\]

\begin{enumerate}[(i)]
	\item \(-5g + 10e + 5f \ne 0\)
	\item \(-5g + 10e + 5f = 0\)
	\item A unique solution.
\end{enumerate}

\subsection{Question}

Determine the values of b so that the following system in unknowns x, y and z has:

\begin{enumerate}[(i)]
	\item No solution.
	\item More than one solution.
	\item A unique solution.
\end{enumerate}

\[3x - 2y + 2z = 5\]
\[2x + y - bz = 5\]
\[5x - by - z = 16\]

\subsection*{Solution}

\[-2 R_1 + 3 R_2 \rightarrow R_2\]
\[-5 R_1 + 3 R_3 \rightarrow R_3\]

\[3x - 2y + 2z = 5\]
\[7y - 3bz - 4z = 5\]
\[-3by + 10y - 18z = 23\]

\[3b R_2 + 7 R_3 \rightarrow R_3\]

\[3x - 2y + 2z = 5\]
\[7y - 3bz - 4z = 5\]
\[- 9b^2z - 12bz - 126z = 15b + 161\]

\chapter{Change of bases and similarity}

\section{Transition matrix}

A transition matrix is a matrix that connects two bases in a vector space.

Supposed that we have two bases \(e_i = (e_1, e_2, \ldots e_n)\) and \(f_i = f_1, f_2, \ldots f_n\), and

\[f_1 = a_{11}e_1 + a_{12}e_2 + \cdots + a_{1n}en\]
\[f_2 = a_{21}e_1 + a_{22}e_2 + \cdots + a_{2n}en\]
\ldots
\[f_n = a_{n1}e_1 + a_{n2}e_2 + \cdots + a_{nn}en\]

\[\begin{bmatrix}
	f_1 \\
	f_2 \\
	\cdots \\
	f_n \\
\end{bmatrix}
=
\begin{bmatrix}
	a_{11} & a_{12} & \cdots & a_{1n} \\
	a_{21} & a_{22} & \cdots & a_{2n} \\
	a_{n1} & a_{n2} & \cdots & a_{nn} \\
\end{bmatrix}
\cdot
\begin{bmatrix}
	e_1 \\
	e_2 \\
	\cdots \\
	e_n \\
\end{bmatrix}\]
\[F = AE\]

The transpose P of the coefficient matrix A is the transition from \(e_i\) to \(f_i\).

On the other hand, the inverse of P (\(P^{-1}\)) is the transition matrix from \(f_i\) to \(e_i\).

\subsection{Example}

Let \({e_i} = {e_1 = (0, 1), e_2 = (0, 1)}\) and \({f_i} = {f_1 = (1, 1), f_2 = (2, 0)}\).

Find:

\begin{enumerate}[(i)]
	\item the transition matrix \(P_1\) from \({e_i}\) to \({f_i}\).
	\item the transition matrix \(P_2\) from \({f_i}\) to \({e_i}\).
	\item and verify that \(P_2 = P^{-1}_1\).
\end{enumerate}

\subsection*{Solution}

\subsection{Theorum}

Let P be the transition matrix from a basis \({e_i}\) to a basis \({f_i}\) in a vector space V. Then, for any vector \(u \in V\), \(P[u]_f = [u]_e\) OR \([u]_f = P^{-1}[u]_e\).

\subsection{Example}

Let \({e_i}\) and \({f_i}\) be as in previous example and let \(u = (2, 3)\). Show that:

\begin{enumerate}[(i)]
	\item \(P[u] = [u]_e\) and
	\item \([u]_f = P^{-1}[u]_e\).
\end{enumerate}

\subsection{Theorum}

Let P be the transition matrix from a bases \({e_i}\) to a bases \({fi}\) in a vector space V. Then for any linear operator T on V, \([T]_f = P^{-1}[T]_eP\).
\end{document}
