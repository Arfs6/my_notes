\documentclass{book}
\usepackage[left=1in,right=1in,top=1in,bottom=1in]{geometry}
\usepackage{amsmath}
\usepackage{amsfonts}
\usepackage{amssymb}
\usepackage{booktabs}
\usepackage{enumerate}
\usepackage{hyperref}
\title{MTH 205 {-} Linear Algebra II}
\author{Abdulqadir Ahmad}
\begin{document}
\maketitle
\tableofcontents

\chapter{Systems of linear equations}

\section{Linear equation}

A linear equation over a field k can be represented by:

\[a_1 x_1 + a_2 x_2 + \cdots + a_n x_n = b\]

Where:

\[i, a_i, b \in k\]

\(x_i\) are unknown variables.

The scalars \(a_i\) are called the coefficient of \(x_i\) respectively and \(b\) is called the constant term of the linear equation.

\section{Solution of a linear equation}

A set of values \(c_i\) is said to be the solution of a linear equation if the linear equation is true when substituting \(c_i\) with \(x_i\).

\[a_1 c_1 + a_2 c_2 + \cdots + a_n c_n = b\]

Where \[c_1, c_2, c_n \in c_i\]

Solutions of linear equations can be categorised into 3:

\begin{enumerate}[a.]
	\item One of the coefficient of \(x_i\) is not equal to 0. Then we can write the solution as:

		\[a_1 x_1 = b - a_2 x_2 + \cdots + a_n x_n \rightarrow x_1 = 1/a_1 \times (b - a_2 x_2 \cdots + a_n x_n)\]

		In this case, we say that the linear equation has many solutions.

	\item All the coefficient of \(x_i\) are 0, but the constant is not 0.

		\[0 x_1 + 0 x_2 + \cdots + 0 x_n = b\]

		where \(b \ne 0\).

		In this case, we say the linear equation is not consistent for there is no solution.

	\item Both the coefficient of \(x_i\) and b are 0.

		\[0 x_1 + 0 x_2 + \cdots + 0 x_n = 0\]

		In this case, the solution of the linear equation can be any set of values in field K.
\end{enumerate}

\section{System of linear equations}

A system of linear equations is a set of m linear equations with n unknowns. For example:

\[a_{11} x_1 + a_{12} x_2 + \cdots + a_{1n} x_n = b_1\]
\[a_{21} x_1 + a_{22} x_2 + \cdots + a_{2n} x_n = b_2\]
\ldots
\[a_{m1} x_1 + a_{m2} x_2 + \cdots + a_{mn} x_n = b_m\]

where

\[a_{ij}, b_i \in K\]

If each \(b_i = 0\), then the system is said to be homogeneous otherwise, the system is non homogeneous.

\section{Solution of a system of linear equation}
A set of values \(c_i = \{c_1, c_2, \cdots c_n\}\) is a particular solution of a system if it satisfies all the linear equations.

The set of all particular solutions is termed the solution set or the general solution.

A homogeneous system always have a solution, I.E. the zero or trivial solution. Any other solution is called a non-trivial solution.

\subsection{Theorem}

Let C be a particular solution of a non-homogeneous system, and let A be the general solution of it's associated homogeneous system, then:

\[c + A = \{c + a \forall a in A\}\]

is the general solution of the aforementioned non-homogeneous system.

\section{Gauss elimination method}

This method involves the elimination of unknowns from succeeding equations by performing two operations:

\begin{enumerate}[I.]
	\item Interchange the equations so that first coefficient of the first isn't 0: \(a_{11} \ne 0\)
	\item For each row in a column, apply the equation:

		\[R_i = -A_{i1}R1 + a_{11}R_i\]

		Where:

		\(A_{ij}\) represents coefficients using matrix element identification.

		\(R_i\) represents a row.

		The goal is to have zeros below the \"diagonal\" of the system. You can achieve this by moving \(a_{ij}\) along the columns.
\end{enumerate}

\subsection{Coefficient and augmented matrix}

An augmented matrix is gotten by writing each equation in a system as a row in a matrix. The unknown variables removed nd a vertical line should be added to represent the equality sign.

A coefficient matrix is an augmented matrix without the constants.


\subsection{Echelon form}

\begin{enumerate}[I.]
	\item The first coefficient of the first row is not equal to zero: \(a_{11} \ne 0\)
	\item All coefficient below the leading diagonal are zeros.
\end{enumerate}

\subsection{Example}

Reduce the following to echelon form and solve:

\[2x + 3y - w + 2z = 6\]
\[3x - y + 2w - z = 3\]
\[5x + 2y + w + z = 8\]

\subsection*{Solution}

\[-3 R_1 + 2 R_2 \rightarrow R_2\]
\[-5 R_1 + 2 R_3 \rightarrow R_3\]
\[2x + 3y - w + 2z = 6\]
\[-11y + 7w - 8z = -12\]
\[-11y + 7w - 8z = -14\]

\[11 R_2 + (-11) \cdot R_3 \rightarrow R_3\]
\[2x + 3y - w + 2z = 6\]
\[-11y + 7w - 8z = -12\]
\[0 = 22\]

\(0 = 22\) the system is inconsistent.

\subsection{Example}

Reduce the following to echelon form and solve:

\[2x + 3y - w + 2z = 6\]
\[3x - y + 2w - z = 3\]
\[x + 2y - 2w + 3z = 4\]
\[4x + 6y - 2w + 4z = 12\]
\[2x - y + w + z = 3\]

\subsection*{Solution}

\[-3 R_1 + 2 R_2 \rightarrow R_2\]
\[- R_1 + 2 R_3 \rightarrow R_3\]
\[-4 R_1 + 2 R_4 \rightarrow R_4\]
\[-2 R_1 + 2 R_5 \rightarrow R_5\]

\[2x + 3y - w + 2z = 6\]
\[- 11y + 7w - 8z = - 12\]
\[y - 3w + 4z = 2\]
\[0 = 0\]
\[- 8y + 4w - 2z = - 6\]

\section{Tutorial Questions}

\subsection{Question}

What condition{(s)} must be placed on e, f and g so that the following system in unknowns x, y and z has:

\begin{enumerate}[i.]
	\item No solution.
	\item More than one solution.
	\item A unique solution.
\end{enumerate}

\[x + 2y - 3z = e\]
\[2x - y + 4z = f\]
\[4x + 3y - 2z = g\]

\subsection*{Solution}
\[-2 R_1 + R_2 \rightarrow R_2\]
\[-4 R_1 + R_3 \rightarrow R_3\]

\[x + 2y - 3z = e\]
\[-5y + 10z = f - 2e\]
\[-5y + 10z = g - 4e\]

\[5 R_2 + (-5) \cdot R_3 \rightarrow R_3\]

\[x + 2y - 3z = e\]
\[-5y + 10z = f - 2e\]
\[0 = -5g + 10e + 5f\]

\begin{enumerate}[i.]
	\item \(-5g + 10e + 5f \ne 0\)
	\item \(-5g + 10e + 5f = 0\)
	\item The system has a free variable, therefore it can't have a unique solution.
\end{enumerate}

\subsection{Question}

Determine the values of b so that the following system in unknowns x, y and z has:

\begin{enumerate}[i.]
	\item No solution.
	\item More than one solution.
	\item A unique solution.
\end{enumerate}

\[3x - 2y + 2z = 5\]
\[2x + y - bz = 5\]
\[5x - by - z = 16\]

\subsection*{Solution}

\[-2 R_1 + 3 R_2 \rightarrow R_2\]
\[-5 R_1 + 3 R_3 \rightarrow R_3\]

\[3x - 2y + 2z = 5\]
\[7y - 3bz - 4z = 5\]
\[-3by + 10y - 18z = 23\]

\[3b R_2 + 7 R_3 \rightarrow R_3\]

\[3x - 2y + 2z = 5\]
\[7y - 3bz - 4z = 5\]
\[- 9b^2z - 12bz - 126z = 15b + 161\]

\begin{enumerate}[i.]
	\item No solution:
		\[- 9b^2z - 12bz - 126z = 0\]
		and
		\[15b + 161 \ne 0\]

	\item More than one solutions:
		\[- 9b^2z - 12bz - 126z = 0\]
		and
		\[15b + 161 = 0\]

	\item Unique solution:
		\[- 9b^2z - 12bz - 126z \ne 0\]
\end{enumerate}

\subsection{Question}

Determine whether each of the following system has a non zero solution.

\begin{enumerate}[i.]
	\item \[x + 2y - 2z = 0\]
		\[x - 8y + 8z = 0\]
		\[3x - 2y + 4z = 0\]

	\item \[x + 3y - 2z = 0\]
		\[2x - 3y + z = 0\]
		\[3x - 2y + 2z = 0\]
\end{enumerate}

\subsection*{Solution}

We can try to find the unique or infinite solutions of each system.

\begin{enumerate}[i.]
	\item \[x + 2y - 2z = 0\]
		\[x - 8y + 8z = 0\]
		\[3x - 2y + 4z = 0\]

		\[-R_1 + R_2 \rightarrow R_2\]
		\[-3R_1 + R_3 \rightarrow R_3\]

		\[x + 2y - 2z = 0\]
		\[-10y + 10z = 0\]
		\[-8y + 10z = 0\]

		\[8R_2 + (-10)R_3 \rightarrow R_3\]

		\[x + 2y - 2z = 0\]
		\[-10y + 10z = 0\]
		\[-20z = 0\]

		Let's check if the system has a unique solution.
		\[-20z = 0 \rightarrow z = 0\]

		Substitute \(z = 0\) into \(-10y + 10z = 0\):
		\[-10y + 10(0) = 0 \rightarrow -10y = 0 \rightarrow y = 0\]

		Lastly, substitute \(y = 0\), \(z = 0\) into \(x + 2y - 2z = 0\):
		\[x + 2(0) - 2(0) = 0 x = 0\]

		Therefore, the system doesn't have a non zero solution.

	\item \[x + 3y - 2z = 0\]
		\[2x - 3y + z = 0\]
		\[3x - 2y + 2z = 0\]

		\[-2R_1 + R_2 \rightarrow R_2\]
		\[-3R_1 + R_3 \rightarrow R_3\]

		\[x + 3y - 2z = 0\]
		\[-9y + 7z = 0\]
		\[-11y + 8z = 0\]

		\[11R_2 + (-9)R_3 \rightarrow R_3\]

		\[x + 3y - 2z = 0\]
		\[-9y + 7z = 0\]
		\[5z = 0\]

		Let's try and find a unique solution:
		\[5z = 0 \rightarrow z = 0\]

		Substitute \(z = 0\) into \(-9y + 7z = 0\):
		\[-9y + 7(0) = 0 \rightarrow -9y = 0 \rightarrow y = 0\]

		Lastly, substitute \(y = 0\), \(z = 0\) into \(x + 3y - 2z = 0\):
		\[x + 3(0) - 2(0) = 0 \rightarrow x = 0\]

		Therefore, the system doesn't have a non zero solution.
\end{enumerate}

\subsection{Question}

What condition{(s)} can you place on b so that the following system has a non-zero solution?
\[bx + 12y = 0\]
\[3x + by = 0\]

\subsection*{Solution}

The condition needs to make the system to have free variables.

But first, let's reduce the system to echelon form.

\[-3R_1 + bR_2 \rightarrow R_2\]

\[bx + 12y = 0\]
\[b^2y - 36y = 0\]

In order to make the system to have free variables, we need to make \(b^2y - 36y = 0\).

Since we're solving for b, we can start by dividing both side by y:
\[b^2 - 36 = 0 \rightarrow b^2 = 36\]

Taking the square root of both side, we'll have:
\[\therefore b = 6\]

\chapter{Change of bases and similarity}

\section{Transition matrix}

A transition matrix is a matrix that connects two bases in a vector space.

Supposed that we have two bases \(e_i = (e_1, e_2, \ldots e_n)\) and \(f_i = f_1, f_2, \ldots f_n\), and

\[f_1 = a_{11}e_1 + a_{12}e_2 + \cdots + a_{1n}en\]
\[f_2 = a_{21}e_1 + a_{22}e_2 + \cdots + a_{2n}en\]
\ldots
\[f_n = a_{n1}e_1 + a_{n2}e_2 + \cdots + a_{nn}en\]

\[\begin{bmatrix}
	f_1 \\
	f_2 \\
	\cdots \\
	f_n \\
\end{bmatrix}
=
\begin{bmatrix}
	a_{11} & a_{12} & \cdots & a_{1n} \\
	a_{21} & a_{22} & \cdots & a_{2n} \\
	a_{n1} & a_{n2} & \cdots & a_{nn} \\
\end{bmatrix}
\cdot
\begin{bmatrix}
	e_1 \\
	e_2 \\
	\cdots \\
	e_n \\
\end{bmatrix}\]
\[F = AE\]

The transpose P of the coefficient matrix A is the transition from \(e_i\) to \(f_i\).

On the other hand, the inverse of P (\(P^{-1}\)) is the transition matrix from \(f_i\) to \(e_i\).

\subsection{Example}
\label{sec:example-1}

Let \({e_i} = {e_1 = (1, 0), e_2 = (0, 1)}\) and \({f_i} = {f_1 = (1, 1), f_2 = (2, 0)}\).

Find:

\begin{enumerate}[i.]
	\item the transition matrix \(P_1\) from \({e_i}\) to \({f_i}\).
	\item the transition matrix \(P_2\) from \({f_i}\) to \({e_i}\).
	\item and verify that \(P_2 = P^{-1}_1\).
\end{enumerate}

\subsection*{Solution}

\begin{enumerate}[i.]
	\item Let's start by finding the content of the matrix.

		\begin{align*}
			(1, 1) &= a_{11}(1, 0) + a_{12}(0, 1) \\
			(2, 0) &= a_{21}(1, 0) + a_{22}(0, 1) \\
	\end{align*}

		We can find \(a_{11}\) and \(a_{12}\) by using linear combination:
		\[(1, 1) = a_{11}(1, 0) + a_{12}(0, 1)\]
		\[(1, 1) = (a_{11}, a_{12})\]
		\[a_{11} = 1, a_{12} = 1\]

		Next, let's find \(a_{21}\) and \(a_{22}\).
		\[(2, 0) = (a_{21} + a_{22})\]
		\[(a_{21} = 2, a_{22} = 0)\]

		\[\begin{bmatrix}
			(1, 1) \\
			(2, 0) \\
		\end{bmatrix}
		=
		\begin{bmatrix}
			1 & 1 \\
			2 & 0 \\
		\end{bmatrix}
		\times
		\begin{bmatrix}
			(1, 0) \\
			(0, 1) \\
		\end{bmatrix}\]

		The transition matrix \(P_1\) is the transpose of the coefficient matrix A.
		\[P_1 = A^T = \begin{bmatrix}
			1 & 2 \\
			1 & 0 \\
		\end{bmatrix}\]
		
	\item \begin{align*}
			(1, 0) &= b_{11}(1, 1) + b_{12} (2, 0) \\
			(0, 1) &= b_{21}(1, 1) + b_{22}(2, 0) \\
			(0, 1) &= b_{12}(1, 1) + b_{22}(2, 0) \\
	\end{align*}

		Let's start by finding \(b_{11}\), b_{12} using linear combination:
		\[(1, 0) = b_{11}(1, 1) + b_{12} (2, 0)\]
		\[(1, 0) = (b_{11} + 2b_{12}, b_{11})\]
		\[b_{11} = 0\]
		\[1 = b_{11} + 2b_{12} \rightarrow 1 = 0 + 2b_{12} \rightarrow b_{12} = 1/2\]

		\[(0, 1) = b_{21}(1, 1) + b_{22}(2, 0)\]
		\[(0, 1) = (b_{21} + 2b_{22}, b_{21})\]
		\[b_{21} = 1\]
		\[0 = b_{21} + 2b_{22} \rightarrow 0 = 1 + 2b_{22} \rightarrow 2b_{22} = -1 \rightarrow b_{22} = -1/2\]
		
		\[B = \begin{bmatrix}
			0 & 1/2 \\
			1 & -1/2 \\
		\end{bmatrix}\]

		Transition matrix is \(B^T\):
		\[B^T = P2 = \begin{bmatrix}
			0 & 1 \\
			1/2 & -1/2 \\
		\end{bmatrix}\]

	\item \[P = \begin{bmatrix}
			1 & 2 \\
			1 & 0 \\
		\end{bmatrix}\]

		\[P^{-1} = \frac{1}{(1 \times 0) - (2 \times 1)} \times \begin{bmatrix}
			0 & -2 \\
			-1 & 1 \\
		\end{bmatrix}\]
		\[P^{-1} = -1/2 \begin{bmatrix}
			0 & -2 \\
			-1 & 1 \\
		\end{bmatrix}
		\rightarrow
		\begin{bmatrix}
			0 & 1 \\
			1/2 & -1/2 \\
		\end{bmatrix} = P2\]
\end{enumerate}

\subsection{Theorem}

Let P be the transition matrix from a basis \(\{e_i\}\) to a basis \(\{f_i\}\) in a vector space V.a Then, for any vector \(u \in V\), \(P{[u]}_f = {[u]}_e\) OR \({[u]}_f = P^{-1}{[u]}_e\).

\subsection{Example}

Let \(\{e_i\}\) and \(\{f_i\}\) be as in Section~\ref{sec:example-1}~(\nameref{sec:example-1}) and let \(u = (2, 3)\). Show that:

\begin{enumerate}[i.]
	\item \(P{[u]}_f = {[u]}_e\) and
	\item \({[u]}_f = P^{-1}{[u]}_e\).
\end{enumerate}

\subsection*{Solution}

Let's start by manually finding \({[u]}_e\) and \({[u]}_f\):
\[{[u]}_e = (2, 3)\]
\[{[u]}_f = (3, -1/2)\]

With this, we can verify our findings.

\begin{enumerate}[i.]
	\item \[P{[u]}_f = {[u]}_e \text{or} {[u]}_e = P{[u]}_f\]
		\[{[u]}_e = \begin{bmatrix}
			1 & 2 \\
			1 & 0 \\
		\end{bmatrix}
		\cdot
		(3, -1/2) = (1 \cdot 3 + 2 \cdot -1/2, 1 \cdot 3 + 0 \cdot -1/2) = (2, 3)\]

	\item \[{[u]}_f = P^{-1}{[u]}_e\]
		\[{[u]}_f = \begin{bmatrix}
			0 & 1 \\
			1/2 & -1/2 \\
		\end{bmatrix}
		\times
		(2, 3)\]
		\[{[u]}_f = (0 \cdot 2 + 1 \cdot 3, 1/2 \cdot 2 + -1/2 \cdot 3) = (3, -1/2)\]
\end{enumerate}

\subsection{Theorem}

Let P be the transition matrix from a bases \(\{e_i\}\) to a bases \(\{f_i\}\) in a vector space V. Then for any linear operator T on V, \({[T]}_f = P^{-1} {[T]}_e P\).

\subsection{Linear operators recap}

Linear operators are functions that takes in vectors from a vector space V and yields vectors in the same vector space.

The matrix representation \({[T]}_e\) of a linear operator T can be gotten by finding the images of all the vectors in the vector space e then:

\begin{itemize}
	\item Switching each vector to it's associated coordinate vectors.
	\item Representing the coordinate vectors as a matrix.
\end{itemize}

\subsection{Example}

Let \(\{e_i\}\) and \(\{f_i\}\) be as in Section~\nameref{sec:example-1}~(\ref{sec:example-1}) and T be the linear operator on \(RR^2\) defined by \(T(x, y) = (x + y, x - y)\), where R is the set of real numbers. Show that \({[T]}_f = P{-1} {[T]}_e P\).

\subsection*{Solution}

Let's start by finding \({[T]}_e\) and \({[T]}_f\):
\[T(e_1) = (1 + 0, 1 - 0) = (1, 1) = e_1 + e_2\]
\[T(e_2) = (0 + 1, 0 - 1) = (1, -1) = e_1 - e_2\]
\[\therefore {[T]}_e = \begin{bmatrix}
	1 & 1 \\
	1 & -1 \\
\end{bmatrix}\]

\[T(f_1) = (1 + 1, 1 - 1) = (2, 0) = (0f_1, f_2)\]
\[T(f_2) = (2 + 0, 2 - 0) = (2, 2) = (2f_1, 0f_2)\]
\[\therefore {[T]}_f = \begin{bmatrix}
	0 & 1 \\
	2 & 0 \\
\end{bmatrix}\]

\[{[T]}_f = P^{-1} {[T]}_e P = \begin{bmatrix}
	0 & 1 \\
	-1/2 & 1/2 \\
\end{bmatrix}
\times
\begin{bmatrix}
	1 & 1 \\
	1 & -1 \\
\end{bmatrix}
\times
\begin{bmatrix}
	1 & 2 \\
	1 & 0 \\
\end{bmatrix}\]

\[{[T]}_f = \begin{bmatrix}
	0 & 1 \\
	1/2 & -1/2 \\
\end{bmatrix}
\times
\begin{bmatrix}
	1 \times 1 + 1 \times 1 & 1 \times 2 + 1 \times 0 \\
	1 \times 1 + (-1) \times 1 & 1 \times 2 + (-1) \times 0 \\
\end{bmatrix}\]

\[{[T]}_f = \begin{bmatrix}
	0 & 1 \\
	1/2 & -1/2 \\
\end{bmatrix}
\times
\begin{bmatrix}
	2 & 2 \\
	0 & 2 \\
\end{bmatrix}\]

\[{[T]}_f = \begin{bmatrix}
	0 \times 2 + 1 \times 0 & 0 \times 2 + 1 \times 2 \\
	1/2 \times 0 + (-1/2) \times 2 & 1/2 \times 2 + (-1/2) \times 2 \\
\end{bmatrix}\]

\[{[T]}_f = \begin{bmatrix}
	0 & 2 \\
	(-1) & 0 \\
\end{bmatrix}\]

\section{Similar matrices}

Let A and B be square matrices and P be an invertible matrix. If:

\[B = P^{-1}AP\]

\itshape{Above expression might be wrongg \ldots}

Then B is said to be similar to A or B is obtained by A similarity transformation.

\subsection{Theorem}
Similarity of matrices is an equivalence relation.

\subsection{Theorem}
Two matrices A and B represents the same linear operator if and only if they are similar to each other.

\section{Diagonalization}

A linear operator is said to be diagonalizable if for some bases \(\{e_i\}\) it is represented by a diagonal matrix. The bases \(\{e_i\}\) is said to diagonalize T.

\subsection{Theorem}

Let A be a matrix representation of a linear operator T. Then T is diagonalizable if and only if their exist an invertible matrix P such that \(P^{-1} A P\) yields a diagonal matrix.

\section{Equivalent matrix}

Two \(m \times n\) matrices A and B over K are said to be equivalent if there exists an m-square invertible matrix Q and an n-square invertible matrix P such that \(B = QAP\).

\section{Tutorial questions}

\subsection{Question}

Let \(\{f_i\} = \{f_1 = (1, 2), f_2 = (2, 3)\}\) and \(\{g_i\} = \{g_1 = (1, 3), g_2 = (1, 4)\}\) 17be basis of \(R^2\).

\begin{enumerate}[i.]
	\item Find the transition matrix P and Q from \(\{f_i\}\) to \(\{g_i\}\) and from \(\{g_i\}\) to \(\{f_i\}\) respectively.
	\item Verify that \(Q = P^{-1}\).
	\item Show that \({[v]}f = P {[v]}g\) for any vector \(v \in R^2\).
	\item Show that \({[T]}g = P^{-1} {[T]}f P\) for each of the operators:

		\begin{enumerate}[a.]
			\item \(T(x, y) = (2x - 3y, x + y)\)
			\item \(T(x, y) = (5x + y, 3x - 2y)\).
		\end{enumerate}
\end{enumerate}

\subsection{Question}

Suppose that \(\{e_1, e_2\}\) is a basis of V and T is a linear operator for which \(T(e_1) = 3e_1 - 2e_2\) and \(T(e_2) = e_1 + 4e_2\).

Let \(\{f_1, f_2\}\) be the basis of V for which \(f_1 = e1 + e2\) and \(f_2 = 2e1 + 3e2\). Find the matrix of T in the basis \(\{f1, f2\}\).

\subsection{Question}

Consider the bases \(\{f_i\} = \{f_1 = 1, f_2 = i\}\) and \(\{g_i\} = \{g_1 = 1 + i, g_2 = 1 + 2i\}\) if the complex field C over the real field R.

\begin{enumerate}[i.]
	\item Find the transition matrices P and Q from \(\{f_i\}\) to \(\{g_i\}\) and from \(\{g_i\}\) to \(\{f_i\}\) respectively.
	\item Verify that \(Q = P^{-1}\).
	\item Show that \({[T]}g = P^{-1} {[T]}f P\) for the operator \(T_z = \bar z\), where \(\bar z\) is the complex conjugate of z.
\end{enumerate}

\subsection{Question}

Suppose that \(\{e_i\}\), \(\{f_i\}\) and \(\{g_i\}\) are basis of V and that P and Q are the transition matrices from \(\{e_i\}\) to \(\{f_i\}\) and from \(\{f_i\}\) to \(\{g_i\}\) respectively. Show that \(P \cdot Q\) is the transition matrix from \(\{e_i\}\) to \(\{g_i\}\).

\subsection{Question}

Show that equivalence of matrices is an equivalence relation.
\end{document}
